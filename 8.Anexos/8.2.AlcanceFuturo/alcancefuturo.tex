El laboratorio de Arduino tiene un amplio alcance para aplicaciones futuras, especialmente en el campo de la automatización, el control de sistemas y la enseñanza de electrónica básica. Los conocimientos adquiridos pueden aplicarse en áreas como la robótica, el internet de las cosas (IoT), y el desarrollo de sistemas embebidos.

A nivel educativo, este tipo de prácticas fomenta el aprendizaje activo y permite a los estudiantes aplicar sus conocimientos en la creación de prototipos reales, lo que a su vez refuerza la comprensión de los principios teóricos de la electrónica. En el ámbito profesional, las aplicaciones prácticas de Arduino permiten diseñar sistemas de control eficientes y económicos, abriendo puertas a innovaciones tecnológicas tanto en el ámbito industrial como en el doméstico.