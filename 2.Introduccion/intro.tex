El presente informe de laboratorio tiene como objetivo explorar el uso de la plataforma Arduino para la construcción y programación de circuitos electrónicos. En particular, se desarrollan una serie de ejercicios prácticos que incluyen el uso de buzzer piezoeléctrico, LEDs y resistencias, aplicando el conocimiento de programación y electrónica básica. El propósito es fomentar la comprensión del funcionamiento de los circuitos lógicos y cómo interactúan con el entorno a través de señales digitales y analógicas.

Los experimentos abarcan desde el encendido de LEDs hasta la generación de tonos con un buzzer, permitiendo a los estudiantes familiarizarse con la programación de Arduino y el montaje de circuitos sencillos. Se busca que los participantes dominen la manipulación de los componentes, el análisis de circuitos, y la interacción con el software de Arduino para programar comportamientos predefinidos.