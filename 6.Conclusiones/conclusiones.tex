A lo largo de la práctica de laboratorio, se logró un dominio fundamental en el uso de Arduino para controlar dispositivos electrónicos, así como en la programación de su microcontrolador. Se pudo evidenciar que la plataforma es accesible tanto para principiantes como para proyectos más avanzados, debido a su flexibilidad y a la vasta cantidad de recursos disponibles.

El uso de circuitos sencillos con LEDs y buzzers permitió a los estudiantes comprender el funcionamiento básico de los componentes electrónicos, así como su programación. Asimismo, la interacción con la comunicación serie facilitó el monitoreo de datos en tiempo real, una habilidad crucial para el desarrollo de proyectos electrónicos más complejos.

Finalmente, se destaca la importancia de la correcta interpretación de los esquemas eléctricos y la adecuada disposición de los componentes en un protoboard, lo que facilita la comprensión teórica y la práctica de conceptos de electrónica y programación.